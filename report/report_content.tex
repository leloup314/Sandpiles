\begin{abstract}
\dots
\end{abstract}

\maketitle

\section{Introduction}
\label{sec:intro}
In this paper we investigate the evolution of sandpiles into states of so-called self-organized criticality (SOC) using
two different models for simulation. Furthermore we analyze the sandpiles' scaling behavior and scaling exponents.

Self-organized criticality describes a special case of the very general concept of criticality. Every system
exhibiting self-organized criticality has in common the occurence of macroscopic scale invariant properties.
In the picture of a real sandpile this can be seen by looking at the average slope of the sandpile,
which stays constant even if the sandpile is enlarged by adding more and more sand to it.
It is notable that this is the case despite the dynamics of a pile of sand is actually governed by complicated
microscopic\footnote{Microscopic in the sense of the size of a sand grain compared to the whole pile.} interactions
of its many constituents, the sand grains.

Such self-organized criticality has been observed before in simulations of sandpiles carried out by Bak, Tang and
Wiesenfeld in the 1980s~\cite{BakTangWiesenfeld}. It is indeed interesting to further analyze such a sandpile model
because it can serve as a toy model for many other similar behaving systems like landslides, rock falls, earth quakes
etc., which all follow from different physical processes, though.
In the mentioned systems signs of self-organized criticality have been observed, not only from simulations but also
from real-world data. %TODO cite

We simulate the sandpiles using a cellular automaton algorithm analog to the approach of Bak, Tang and Wiesenfeld,
which is commonly referred to as the \enquote{BTW model}. Scaling exponents are determined and compared to the
results of a customized sandpile model.

Section~\ref{sec:theory} starts with an overview on the theory behind self-organized criticality and in particular
explains the scaling behavior governed by the scaling exponents.
In section~\ref{sec:experiment} our implementation of the sandpile cellular automata is shown as well as
the used methods for extracting the relevant parameters from the generated data.
The results are presented in section~\ref{sec:results} and discussed in section~\ref{sec:analysis}.


\section{Theory}
\label{sec:theory}

\subsection{Self-organized criticality}
\label{sec:th:SOC}
To understand the concept of self-organized criticality consider first a system that exhibits a critical state,
like the Ising model. The critical state in the Ising model corresponds to a phase transition where vanishing
magnetization evolves into spontaneous magnetization. This happens at a specific temperature, the critical temperature
$T_{\mathrm{crit}}$. To obtain a system in the critical state, the temperature has to be precisely tuned to the critical
temperature.

The criticality of the system leads to scale invariant properties. For instance clusters of the same spin direction
will form and the size distribution of these clusters follows a simple power law. A power law distribution implies
scale invariance, where scale invariance means that the dynamics of the considered system does not change if the system
is scaled by some global factor. In particular the observables' distribution functions keep the same form.

The crucial difference of this definition of criticality and the criticality observed in the case of the sandpile model
is that sandpiles will evolve into a critical state without tuning of any critical parameter.
If enough sand is deposited into a sandbox or, respectively, enough iteration steps of the computer model are performed
it reaches an equilibrium state by itself with a fixed average slope. This self-organization into a critical state
allways happens, independently of the actual detailed model parameters.\footnote{Note that the slope can still depend
on these parameters.} Thus this phenomenon is called \enquote{self-organized criticality}.

\subsection{Scaling exponents}
\label{sec:th:scaling}
If there is scale invariance in a system it can be directly observed in the distribution functions
of the observables of the system as indicated above. Consider some observabe $\hat{O}$ that can be measured in the
present system. If this observable is measured many times and histogrammed one obtains a distribution according to the
corresponding probability density function $P^{O}(o)$.
If now the system is scaled by a factor $\lambda$ it must hold
\begin{equation}\label{eq:scalingCondition}
P^{O}(\lambda o) = f(\lambda) \times P^{O}(o)
\end{equation}
due to the assumed scale invariance.

As scale invariance is suspected as a main property of SOC models in general, one is mainly interested in studying
these distribution functions to proof this. To do this a theoretical prediction is necessary first.
In the simplest approximation the distribution function of an observable $\hat{O}$ in a scale invariant SOC model
is assumed to be a simple power law
\begin{equation}\label{eq:simpleScaling}
P^{O}(o) \sim o^{-\rho}
\end{equation}
in order to fulfill Eq.\,\eqref{eq:scalingCondition} as $(\lambda o)^{-\rho}\sim o^{-\rho}$.
Here $\rho$ is called the \emph{scaling exponent} or \emph{critical exponent} of the observable $\hat{O}$.

However, in a real implementation of the sandpile model or any other SOC model one has to deal with finite system sizes
since infinitely large systems can neither be simulated nor do they exist in nature.
A finite system size limits the validity of Eq\,\eqref{eq:simpleScaling} because for instance avalanche sizes on a
sandpile are naturally limited to the size of the sandpile itself. Thus a more sophisticated description of scaling
has to be used, so-called \emph{finite-size scaling} as described in \cite{SOC-book}:
\begin{equation}\label{eq:finiteSizeScaling}
P^{O}(o) = a o^{-\rho} G^{O}\left(\sfrac{o}{o_c(L)}\right), \quad o_c(L)=b L^D
\end{equation}
The additional scaling function $G^{O}$ accounts for limited size corrections to the simple power law scaling
by modifying the probability density for values of $o$ in the order of the limiting \emph{characteristic scale}
$o_c(L)$. This characteristic scale where limitations eventuate can itself depend on the system size $L$,
e.g. the maximum area coverage of a two dimensional lattice is $L^2$. Thus the characteristic scale is approximately
assumed to behave as $o_c(L)=b L^D$ for any observable, where $b,D$ depend on $\hat{O}$.

One possibility to obtain the scaling exponent $\rho$ (and also $D$) from the measured distribution functions
is via their moments $\langle P^n\rangle$.
\dots%TODO


\section{Experimental Methods}
\label{sec:experiment}

\subsection{Cellular Automata}
\label{sec:cellularAutomata}
To realize the simulation of sandpiles it is convenient to construct a cellular automaton.
A cellular automaton is an algorithm that takes values on a discrete $n$-dimensional lattice and iteratively
performs updates on all lattice sites with each updated value only depending on the old value and the values of its
nearest neightbours (and possibly random numbers). Additionally the lattice sites can be periodically perturbed,
for instance increasing the value of a randomly picked lattice site.
\dots%TODO?

\subsection{Implementation of sandpile cellular automata}
\label{sec:sandpileImplementation}
In general a cellular automaton for a sandpile in $\mathcal{N}$ dimensions works on an $\mathcal{N}-1$ dimensional
lattice where each lattice site stands for a stack of sand grains in the left dimension. That means for example that
a real sandpile, which is three dimensional, is simulated on a two dimensional lattice reflecting the ground as $x$-$y$
plane. The entry on each site then characterizes how many grains of sand are stacked in the $z$ direction.

It can be either chosen to store the actual amount of grains as the entry or the slope with respect to the vicinity
of the stack. Both choices enable for sufficient characterization and simulation of the microscopic dynamics of the
sandpile. Chosing the slope entrys, however, has the drawback that the model will then be non isotropic anymore, as will
become clear in the following, whereas storing the actual heights can make the model implementation a bit more complex.

Within the scope of this paper we have implemented two different cellular automata that are based on these two different
choices. The first approach reflects the BTW model.
The BTW model by Bak, Tang and Wiesenfeld is the first model investigating SOC in sandpile dynamics. It uses a lattice
that contains slope values and alternatingly performs a \emph{driving} and a \emph{relaxation} of the lattice:

\textbf{Driving} generally describes the active perturbation of the lattice. For a real sandpile the perturbation
intuitively consists of adding a grain of sand to the pile at a random position. Since the BTW model deals with 
slopes the addition of one grain of sand is achieved by the following algorithm~\cite{BakTangWiesenfeld}:
\begin{itemize}
\item \dots%TODO
\end{itemize}
\dots%TODO

\textbf{Relaxation}
\dots
\begin{itemize}
\item \dots%TODO
\end{itemize}
\dots%TODO
%TODO mention non-isotropic...

The second approach to the sandpile simulations uses the local number of sand grains as lattice entries. Thus it can
circumvent the problem of the BTW model to be non isotropic because the slopes can be directly calculated from the
heights in every direction.
\dots%TODO

\subsection{Extraction of scaling exponents}
\label{sec:extractCritExp}


\section{Results}
\label{sec:results}


\section{Analysis}
\label{sec:analysis}


\section{Conclusion}
\label{sec:conclusion}


\section{Suggestions for Further Research}
\label{sec:further_research}




\begin{acknowledgments}
\end{acknowledgments}


\appendix*

\section{}
\label{sec:}
\subsection{}
\dots
